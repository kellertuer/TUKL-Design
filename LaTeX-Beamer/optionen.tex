%
% THE BEER-WARE LICENSE (Rev. 42):
% Ronny Bergmann <bergmann@mathematik.uni-kl.de> wrote this file. As long as
% you retain this notice you can do whatever you want with this stuff. If we
% meet some day, and you think this stuff is worth it, you can buy me a beer
% or coffee in return.
%
% This file is just to get started - You need the corresponding Logo
%
\documentclass[german,10pt,xcolor=colortbl,compress
%,draft
]{beamer}
\usepackage{xunicode}
\usepackage[OT1]{fontenc}
\usepackage{calc}
\usepackage[ngerman]{babel} % Neue Rechtschreibung
\usepackage{amsmath,amsthm,amssymb,euscript} % AMS-LaTeX  
\usepackage{enumerate,graphicx,booktabs}
\usepackage[TeX]{listings}
\lstset{basicstyle=\small\ttfamily}
\RequirePackage[no-math]{fontspec} 
\RequirePackage{xltxtra}
\defaultfontfeatures{Mapping=tex-text} % converts LaTeX specials (``quotes'' --- dashes etc.) to unicode
% Load Them
\usetheme[meta=false,navigation=true,reducedframetotal%,
]{TUKL}
% Meta alternative in XeLaTeX
\setsansfont[Ligatures={Common}]{Arial}
%
\setbeamertemplate{navigation symbols}{}
\title{\LaTeX-Beamer Theme im Stil der TU Kaiserslautern}
\subtitle{Verwendung und Optionen dieses Themes}
\date[]{\today}
\author[R. Bergmann]{Ronny Bergmann}
\institute[]{AG Bildverarbeitung\\FB Mathematik\\TU Kaiserslautern}
\hypersetup{colorlinks=true, linkcolor=tuklblue, urlcolor=blue!50!black}

\renewcommand{\theSecondLogo}{\includegraphics[width=.15\paperwidth]{logos/Logo_FKZM_rgb.jpg}}

\begin{document}
	\maketitle
	\begin{frame}{Inhaltsverzeichnis}
		\tableofcontents
	\end{frame}
	\section{Einrichten}
	\begin{frame}[fragile]{Benötigte Pakete}
		folgende Pakete werden von dem Theme benötigt und per \lstinline!\RequirePackage! eingebunden:
		\begin{itemize}
			\item \lstinline!ifthen!
			\item \lstinline!pdftexcmds!
			\item \lstinline!calc!
			\item \lstinline!TikZ!
		\end{itemize}\vspace{\baselineskip}
		sowie bei XeLaTeX zusätzlich
		\begin{itemize}
			\item \lstinline!fontspec!
		\end{itemize}\vspace{\baselineskip}
		Bei der Option \lstinline!meta=true! muss außerdem die Schriftart Meta im System installiert sein.
	\end{frame}
	\begin{frame}{Einbinden des Themes}
		\begin{itemize}
			\item die 4 Dateien
			\begin{itemize}
				\item \lstinline|beamercolorthemeTUKL.sty|
				\item \lstinline|beamerinnerthemeTUKL.sty|
				\item \lstinline!beameroutertheme.sty!
				\item \lstinline|beamerthemeTUKL.sty| 
			\end{itemize}
			\item das TU-Logo als \lstinline!TU-KL-RGB.jpg! in den Ordner \lstinline!logos/! legen
			 im gleichen Verzeichnis oder im \TeX-Tree lagern
			\item nach den Paketen mit \textbackslash\lstinline|usetheme[<Optionen>]\{TUKL\}| das Paket laden
			\item Am einfachsten: Mit einem der Beispiele (pdflatex oder \XeLaTeX) loslegen
		\end{itemize}
	\end{frame}
	\section{Optionen}
	\subsection*{}
	\begin{frame}{Optionen}{Fachbereichsfarben mit den RGB-Farben aus dem Coorporate Design}
				\begin{tabular}{ll}
					\toprule
					\lstinline|FB=none| (Standardwert)& Standardfarben der TU (wie hier)\\
					\lstinline|FB=Architektur| & \colorbox{tuklarchitektur}{\color{white}Architektur}\\
					\lstinline|FB=Maschinenbau| & \colorbox{tuklmaschinenbau}{\color{white}Maschinenbau und Verfahrenstechnik}\\
					\lstinline|FB=Bauignenieurwesen| & \colorbox{tuklbauingenieurwesen}{\color{white}Bauingenieurwesen}\\
					\lstinline|FB=Mathematik| & \colorbox{tuklmathematik}{\color{black}Mathematik}\\
					\lstinline|FB=Biologie| & \colorbox{tuklbiologie}{\color{white}Biologie}\\
					\lstinline|FB=Physik| & \colorbox{tuklphysik}{\color{white}Physik}\\
					\lstinline|FB=Chemie| & \colorbox{tuklchemie}{\color{white}Chemie}\\
					\lstinline|FB=Raumplanung| & \colorbox{tuklraumplanung}{\color{white}Raum- und Umweltplanung}\\
					\lstinline|FB=Elektrotechnik| & \colorbox{tuklelektrotechnik}{\color{white}Elektrotechnik und Informationstechnik}\\
					\lstinline|FB=Sozialwissenschaften| & \colorbox{tuklsozialwissenschaften}{\color{white}Sozialwissenschaften}\\
					\lstinline|FB=Informatik| & \colorbox{tuklinformatik}{\color{white}Informatik}\\
					\lstinline|FB=Wirtschaftswissenschaften| & \colorbox{tuklwirtschaftswissenschaften}{\color{white}Wirtschaftswissenschaften}\\
					\bottomrule
				\end{tabular}
	\end{frame}
	\begin{frame}{Schriftart}{nur XeLaTeX}
		\begin{itemize}
			\item \lstinline|meta=true| oder \lstinline|usemeta| lädt mit dem Paket \lstinline|fontspec| die Schriftart Meta, so diese im System geladen ist
			\item Alternativ: Zeilen im \XeLaTeX-Beispiel laden die Schriftart Arial
			\item Achtung: \lstinline|fontspec| ist ein \XeLaTeX-Paket
			\item \lstinline|meta=false| oder \lstinline|nometa| lädt weder das Paket \lstinline|fontspec| noch eine Schriftart.
			\item Standardwert: \lstinline|meta=false| (funktioniert standardmäßig also auch mit pdf\LaTeX)
		\end{itemize}
	\end{frame}
	\begin{frame}{Gesamtfolienanzahl und Navigation}
		\begin{itemize}
			\item \lstinline|frametotal=true| bzw. \lstinline|displayframetotal| Zeige hinter der Foliennummer in der Fußzeile die Gesamtfolienanzahl. Beispiel: 5/\inserttotalframenumber
			\item \lstinline|frametotal=false| bzw. \lstinline|hideframetotal| Zeige die Gesamtanzahl nicht (wie in diesem Foliensatz)
			\item Standardwert: \lstinline|frametotal=true|
			\item analog: \lstinline|displayframetotal| und \lstinline|hideframetotel|
			\item zusätzlich: \lstinline|reducedframetotal| - Effekt: siehe diese Folien.
		\end{itemize}
		\vspace{\baselineskip}
		\begin{itemize}
			\item \lstinline|navigation=true| bzw. \lstinline|displaynavigation| Zeige in der Kopfzeile eine Navigation an (wie in diesm Foliensatz)
			\item \lstinline|navigation=false| bzw. \lstinline|hidenavigation| Zeige keine Navigation an
			\item Standardwert: \lstinline|navigation=true|
		\end{itemize}
	\end{frame}
	\begin{frame}[plain]{Leere Folien}
		Wird einmal mehr Platz benötigt, kann mit der option \lstinline|[plain]| des \lstinline|frame| eine komplett leere Folie erzeugt werden, wie diese hier.
	\end{frame}
	\section{Befehle}
	\subsection*{}
	\begin{frame}[fragile]{Titelseite}
		Der Befehl \lstinline|\maketitle| erzeugt eine Titelfolie.
		
		Er kann entweder in einem \lstinline|frame| aufgerufen werden, dann wird der Stil des Frames verwendet.
		
		Außerhalb eines Frames wird ein Titelframe erzeugt, der den Standardkopf hat, aber keine Fußzeile (siehe Titelfolie dieser Folien).
	\end{frame}
	\begin{frame}[fragile]{Ein zweites Logo auf der Titelseite}
		Der Befehl \lstinline|\theSecondLogo| setzt ein neues Logo, indem man diese umdefiniert.
		
		Dabei wird in dem Befehl die gesamte Grafik eingebunden; in diesen Folien etwa das Logo des Felix-Klein-Zentrums mittels
		
		\begin{lstlisting}
\renewcommand{\theSecondLogo}%
	{\includegraphics[width=.15\paperwidth]{logos/Logo_FKZM.jpg}}
		\end{lstlisting}		
	\end{frame}
	\begin{frame}[fragile]{Der Fachbereichsschriftzug im Kopf}
		Der Befehl \lstinline|\theDept| wird mit der Fachbereichs-Option \lstinline|FB=| auf den Namen des Fachbereichs gesetzt.
		\\[\baselineskip]
		Mit \lstinline|\renewcommand{\theDept}{}| kann der Text auf leer zurückgesetzt werden.
	\end{frame}
	\begin{frame}{Links}
		\begin{itemize}
			\item Downloadbereich der Presse-Abteilung\\ \href{http://www.uni-kl.de/universitaet/pr-marketing/download-und-vorlagen/}{www.uni-kl.de/universitaet/pr-marketing/download-und-vorlagen/}
			\item Userguide zu \LaTeX-Beamer \href{ftp://ftp.dante.de/tex-archive/macros/latex/contrib/beamer/doc/beameruserguide.pdf}{dante.de/tex-archive/macros/latex/contrib/beamer/doc/beameruserguide.pdf}
			\item Aktuellste Version dieses Paketes auf \href{http://github.com}{github.com}:\\
			\href{https://github.com/kellertuer/TUKL-Design/archive/master.zip}{https://github.com/kellertuer/TUKL-Design/archive/master.zip}
		\end{itemize}
	\end{frame}
\end{document}